% Preamble
% ---
\documentclass{article}

% Packages
% ---
\usepackage{amsmath} % Advanced math typesetting.
\usepackage[utf8]{inputenc} % Unicode support (Umlauts etc).
\usepackage{hyperref} % Add a link to your document.
\usepackage{listings} % Source code formatting and highlighting.
\usepackage{xfrac} % For slanted fractions.
% ---

\newcommand\p[1]{\left( #1 \right)}
\newcommand\dice[2]{#1\space\text{d}#2}
\renewcommand{\d}{\text{d}}

\begin{document}

From \href{https://towardsdatascience.com/modelling-the-probability-distributions-of-dice-b6ecf87b24ea}
{Modelling the probability distributions of dice | by Tom Leyshon | Towards Data Science}

\begin{align}
    P(n\d s=x) & =\p{\sum_{k=0}^{\left\lfloor \frac{x-n}{s}\right\rfloor}\p{-1}^k\frac{n!}{\p{n-k}!k!}\frac{\p{x-sk-1}!}{\p{x-sk-n}!\p{n-1}!}}\p{\frac{1}{s}}^n \\
    P(n\d s=x) & =\p{\frac{1}{s}}^n\sum_{k=0}^{\left\lfloor \frac{x-n}{s}\right\rfloor}\p{-1}^k{n \choose k}\frac{\p{x-sk-1}!}{\p{x-sk-n}!\p{n-1}!}
\end{align}

\[
    n! = \prod_{k=1}^{n}k
\]

\begin{align}
    \d s        & = 1\d s                                                  \\
    n\d 1       & = n                                                      \\
    \p{n+m}\d s & = n\d s + m\d s                                          \\
    \d \p{sz}   & = \p{\d s - 1} + s\p{\d z - 1} + 1                       \\
    P\p{\d s=x} & = \begin{cases}
                        \sfrac{1}{s}, & \text{for}\; x \in \left[ 1, s \right] \\
                        0,            & \text{otherwise}
                    \end{cases} \\
    n\d 2 - 1   & \sim \text{Binomial}\p{n-1, \sfrac{1}{2}}
\end{align}

\begin{align}
    P(X+Y=z) & =\sum_{x=-\infty }^{\infty }P(X=x)P(Y=z-x)
\end{align}

\begin{align}
    P\p{\text{highest}\p{1, n\d s}=x} & = \begin{cases}
                                              \frac{x^n-\p{x-1}^n}{s^n}, & 1\leq x\leq s    \\
                                              0,                         & \text{otherwise}
                                          \end{cases}                      \\
    P\p{\text{lowest}\p{1, n\d s}=x}  & = \begin{cases}
                                              \frac{\p{s-x+1}^n-\p{s-x}^n}{s^n}, & 1\leq x\leq s    \\
                                              0,                                 & \text{otherwise}
                                          \end{cases} \\
    \text{highest}\p{n, n\d s}        & = \text{lowest}\p{n, n\d s} = n\d s
\end{align}

\end{document}